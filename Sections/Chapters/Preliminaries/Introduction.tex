\chapter*{Introduction}\addcontentsline{toc}{chapter}{Introduction}
According to the Danish Ministry of Health, 10.4\% of the danish BNP is spent on health and health related expenses \cite{sum:sundhedsudgifter}. Most of the spending goes to hospitalization.

By January 2017, there were 1.095.172 danish citizens aged 65 and above, which amounts to 19.1\% of the population \cite{aeldrepdf}. Among the elderly, falling can be a severe problem. About 1/3 of elderly has a fall accident at least once per year. Every year, these falling accidents results in deaths or other medical complications, due to help not arriving in time. Furthermore, out of the citizens who actually has a fall accident, half of them will likely have another fall accident in the future.\\
Even though only about 10\% of these accidents result in serious harm, these account for 20-30\% of all accident related hospitalizations \cite{SundhedFald}. These accidents are expensive for the danish government. Even if only the accidents that results in death are taken into account, this amounts to a cost of 4.8B DKK every year \cite{faldkost}.

Many different systems exist for dealing with fall accidents. These are explored in section \ref{sec:existing_solutions}. Many of these focus on necklace based systems, and mostly on passive fall detection - meaning that the user needs to click a button to call for help. Some systems detect fall accidents based on input from a motion sensor, or based on basic voice detection.

This project aims to develop a smarter solution than existing ones. This is achieved by automating certain parts of such a solution. Many current solutions rely on either a call center, a dedicated device such as a necklace, or both. We have therefore attempted to develop a web based solution, by trying to integrate everyday technologies, such as smartphones into a smart solution \cite{smartgrow}.


\section*{Initiating problem}
The following initial problem has been defined:

\begin{center}
    \textit{What is required of a system, that can assist (elderly) people after they have had a falling accident?}
\end{center}

Furthermore, the following sub-questions has been defined:
\begin{itemize}
    \item Who is at risk of having a fall accident?
    \item What existing solutions already try to solve the problem, and how?
    \item Which platforms (both hardware and software) would be best suited for the problem?
    \item What other technologies, such as IoT devices, can be utilized in this context?
\end{itemize}