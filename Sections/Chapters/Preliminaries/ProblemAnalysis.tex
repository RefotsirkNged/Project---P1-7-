\label{preliminaries:problemanalysis}

In this chapter an analysis of the problem domain of fall incidents is presented. Firstly the problem as a whole is described, where after the group of citizens in risk of falling are defined, and cases are set up further understand the target group.

\section{Fall incidents}
According to \textit{sundhed.dk} \cite{SundhedFald}, a site that is administrated by the danish government, 1/3 of danish citizens over the age of 65 has a fall incident at least once a year.

% What is a fall?
In \textit{Sundhedsstyrrelsen} \cite{Sundhedsstyrrelsen:Faldpatienter}, a fall incident is defined as follows:

\begin{center}
    \textit{An unintended action that results in a citizen ending on the floor or other lower floor.}
\end{center}

Falling incidents are further separated into three different categories:
\begin{itemize}
    \item Tripping and other accidental accidents
    \item Unexplained fall with and without dizziness
    \item Fall with loss of consciousness
\end{itemize}


The first category \textit{Tripping and other accidental accidents} are tripping accidents that happens to both healthy and unhealthy citizens. They are common tripping accidents that can happen to anybody.

The second category \textit{Unexplained fall with and without dizziness} are tripping accidents that happens for unexplained reasons, not including losing  consciousness.

The third category \textit{Fall with loss of consciousness} are tripping accidents, that are caused by loosing consciousness.

We are focusing on citizens who are not capable of getting up after a fall incident, and specifically this includes citizens who loose consciousness. We do not limit ourselves to any of the three categories, as all are potential users of a falling detection system.


It is not always the fall itself that is the most dangerous for the citizen. Hip fracture often happens when the citizen fails to get up again without help \cite{CekuraFald}. The citizen should therefore always try to get help from others, either by calling for help, or by trying to make noise so that they can be found.

Falling accidents among citizens has severe financial costs for society. It has been estimated that falling accidents make up for two percentage of the public health care expenses. In Denmark that translates to two billion dkr. every year, whereas 800 million is expenses from the municipalities them self.

An example case could be, when a citizen has a falling accident and the citizen gets a hip fracture. The medical expenses has been estimated to 200.000 kr. for hospitalization, home care and medicine for that citizen alone. Since the amount of falling accidents increase every year, then the expenses will also increase \cite{MagasinetSundhed:Pris}.

\subsection{Risk groups}
There are multiple factors that can increase the risk of a fall accident. \textit{Sundhedsstyrelsen} \cite{FaldArtikel} lists the following factors (translated from danish):

\begin{itemize}
    \item Being a woman.
    \item Being older than 80 years.
    \item If you have had a previous fall accident.
    \item Cognitive failure, dementia and delirium.
    \item Reduced mobility.
    \item Dizziness and decreased balance.
    \item Decreased vision.
    \item Low BMI.
    \item Use of four or more medicaments.
    \item Psychiatric drugs.
    \item Three or more chronic disorders.
    \item Use of walking aids.
    \item Osteoarthritis.
    \item Reduced muscle strength.
\end{itemize}

From this we get that the citizen in risk of having a falling incident are the elderly citizens, with or without varying degree of physical disabilities. From all these, the target user group for this project are those who cannot get up by them self, this can be both if they are unconscious or any other reason, such as them being too weak or in too much pain.

\section{Scenarios} \label{preliminaries:problemanalysis:limitproblem}

We have limited out problem area to handling the problems that arise after the citizens have fallen, where they are unable to get up by themselves. From research we have found that while a lot is put into fall prevention, there is not a lot about what to do after they have fallen. We have limited ourselves to focus on how to assists the citizen after they have fallen. To better understand the problem, we have set up five scenarios that reflects what could happen when a citizen has fallen.

\subsection{Case 1:}
Mr. Jensen was at home in his kitchen, when he stumbles and falls on the ground. In the fall he got a few bruises, but otherwise no physical damage. He was able to get up by himself, without further problems.

\subsection{Case 2:}
Mr. Jensen was getting out of bed, but stumbles over his slippers, and ends on the ground. He cannot come up, and have no way he can get in contact with another person, and no phone within reach. If he is not able he can drag himself to a phone, there is not much he can do unless he manages to regain enough energy to try and get up by him himself, though he puts himself at risk of falling again while trying.

\subsection{Case 3:}
Mr. Jensen was walking through his home when he stumbled over the edge of the door frame. He fell and hit his head on impact, knocking him unconscious. He was unable to do anything to call for help, even though he had a phone on him.

\subsection{Case 4:}
Mr. Jensen was working outside in the garden, when he fell over a stone. He was unable to get up by himself, and had no choice but calling for help. Being outdoor means that a neighbor heard him shortly after. But had he not been heard, he could have done nothing since he had no phone on him while outside.

\subsection{Case 5:}
Mr. Jensen was walking in his garden when he fell over a tool he had forgotten to return to his shed. He hit his head on impact, knocking him unconscious. There was no option for him to call for help. He didn't bring a phone on him either.

\subsection{Conclusion}
Of the five cases we will focus on cases 2-5, as they fit the limitations that we have set of people unable to get up by themselves. Since the person in case 1 is able to get up by himself, that then means case 1 is outside our scope.

In the next chapter, different existing solutions that can handle some of or all these cases, has been evaluated, including the one used by the danish health care.