\label{preliminaries:problemanalysis}
This chapter contains an analysis of the problem domain, fall accidents and fall detection. It contains an analysis of the types of fall accidents, statistics on how many accidents happen, and of the different risk groups for which the fall incidents happen. The chapter is concluded with scenarios describing the kinds of accidents targeted by the solution developed through this project.

\section{Fall accident}
According to \textit{sundhed.dk} \cite{SundhedFald}, a site that is administrated by the danish government, 1/3 of danish citizens over the age of 65 has a fall accident at least once a year.

In \textit{Sundhedsstyrrelsen} \cite{Sundhedsstyrrelsen:Faldpatienter}, a fall accident is defined as follows:

\begin{center}
    \textit{An unintended action that results in a citizen ending on the floor or other lower floor.}
\end{center}

Falling accidents are further separated into three different categories:
\begin{itemize}
    \item Tripping and other accidents
    \item Unexplained fall with and without dizziness
    \item Fall with loss of consciousness
\end{itemize}

The first category \textit{Tripping and other accidental accidents} are tripping accidents that happens to both healthy and unhealthy citizens. They are common tripping accidents that can happen to anybody.

The second category \textit{Unexplained fall with and without dizziness} are tripping accidents that happens for unexplained reasons, not including losing consciousness.

The third category \textit{Fall with loss of consciousness} are tripping accidents, that are caused by loss of consciousness.

This project is focused on citizens who are not capable of getting up after a fall incident, and specifically this includes citizens who lose consciousness. None of the categories are excluded, as all are potential users of a falling detection system.

It is not always the fall itself that is the most dangerous for the citizen. Hip fracture often happens when the citizen fails to get up again without help \cite{CekuraFald}. The citizen should therefore always try to get help from others, either by calling for help, or by trying to make noise such that they can be found.

Falling accidents among citizens has severe financial costs for society. It has been estimated that falling accidents make up for two percentage of the public health care expenses. In Denmark that translates to 4.8b DKK\cite{faldkost} every year.

An example case could be, when a citizen has a falling accident and the citizen gets a hip fracture. The medical expenses has been estimated to 200.000 kr. for hospitalization, home care and medicine for that citizen alone. Since the amount of falling accidents increase every year, the expenses will also increase \cite{MagasinetSundhed:Pris}.

\subsection{Risk groups}
There are multiple factors that can increase the risk of a fall accident. \textit{Sundhedsstyrelsen} \cite{FaldArtikel} lists the following factors (translated from danish):

\begin{itemize}
    \item Being a woman.
    \item Being older than 80 years.
    \item If you have had a previous fall accident.
    \item Cognitive failure, dementia and delirium.
    \item Reduced mobility.
    \item Dizziness and decreased balance.
    \item Decreased vision.
    \item Low BMI.
    \item Use of four or more medicaments.
    \item Psychiatric drugs.
    \item Three or more chronic disorders.
    \item Use of walking aids.
    \item Osteoarthritis.
    \item Reduced muscle strength.
\end{itemize}

From this can be seen that the citizens in risk of having a falling accident are the elderly citizens, with or without varying degrees of physical disabilities. From all these, the conclusion is that the target user group for this project are mostly elderly citizens who cannot get up by themselves after an accident, this can be both if they are unconscious or any other reason, such as them being too weak or in too much pain.

\section{Scenarios} \label{preliminaries:problemanalysis:limitproblem}
In the previous section, the problem was limited to handling the problems that arise after citizens has fallen, and specifically to cases where they are unable to get up by themselves. To better understand the problem, the following sections set up five scenarios that reflects on what could happen when a citizen has fallen.

\subsection{Case 1:}
Mr. Jensen was at home in his kitchen, when he stumbled and fell to the ground. During the fall he got a few bruises, but otherwise no physical damage. He was able to get up by himself, without further problems.

\subsection{Case 2:}
Mr. Jensen was getting out of bed, but stumbles on his slippers, and falls on the ground. He cannot get up, and has no way to get in contact with another person due to not being able to reach a phone. Unless he can manage to crawl towards a phone, or otherwise get help by himself, he will not be able to get up again.

\subsection{Case 3:}
Mr. Jensen was walking through his home, but stumbled on the edge of the door frame. He fell and hit his head on impact, knocking him unconscious. He was unable to do anything to call for help, even though he had a phone on him.

\subsection{Case 4:}
Mr. Jensen was working outside in his garden, when he stumbled on a stone. He was unable to get up by himself, and had no choice but yelling for help. Being outdoor means that a neighbor heard him shortly after. But had he not been heard, he could have done nothing since he had no phone on him while outside.

\subsection{Case 5:}
Mr. Jensen was walking in his garden when he fell over a tool he had forgotten to return to his shed. He hit his head on impact, knocking him unconscious. There was no option for him to call for help. He did not bring a phone either.

\subsection{Conclusion}
Of the five cases we will focus on cases 2-5, as they fit the limitations that we have set for people unable to get up by themselves. Since the person in case 1 is able to get up by himself, the case is outside our scope.

In relation to the initial problem, this chapter answers one of the sub-questions, "Who is in risk of having a fall accident?".
The risk group mainly consists of elderly citizens, or citizens otherwise at risk of falling, and they are the target audience for this report.

In the following chapter, different existing solutions that can handle some of, or all these cases, has been evaluated, including the one used by danish health care.