\label{problemdefinition}

After the initial analysis, in this chapter we have summarized the conclusions of the previous chapters, and defined the problem statement.

In chapter \ref{preliminaries:problemanalysis} we found that elderly people have falling accidents, and not always are able to get up without help.
In chapter \ref{preliminaries:existingsolutions} we found that \todo{What did we find?}
In chapter \ref{chap:technologies} we came to the conclusion that we will continue working with smarthones and personal intelligent assistants. 

\iffalse

After having done the initial analysis of the subject area, we have defined the problem statement.

In the problem analysis in chapter \ref{preliminaries:problemanalysis} we investigated different falling accident among elderly people. We have researched different scenarios of what happens when a citizen had a falling accident, and limited the project to the cases where the citizen cannot come up by themselves, both with and without consciousness, from this we have concluded that a system to get help is needed.

With an initial idea of the problem, we examined some different existing solutions in chapter \ref{sec:existing_solutions}, including one in use in a danish municipality. These all implement a wide array of different technologies to try and solve the problem. With that as a base we then gave an overview of the many different technologies that could be used for fall detection, this can be found in Chapter \ref{chap:technologies}.

We found that the existing solutions are not sufficient. Many of the existing systems are either too expensive, or require more input from the citizen than might be possible in the situation. The systems are also limited to their specific use case only, with no room to expand on those solutions. This is also reflected in the statistics provided in chapter \ref{preliminaries:problemanalysis}.

%Skrivet a carsten 
The current solutions don't have the ability to expand upond, where our solution should be able to, this is why we are making this as a web based solution, this would allow to easy management of setting and profile from any device. Expanding the system to use new devices, would be to make a Web-API that would allow any device to talk to the system and expand what information could be collect about person to help them.

In chapter \ref{chap:technologies} we limited our self to a focus on smartphones and personal assistants, but with the option to integrate other smart devices in the future.

\fi

Based on that we have come up with the following problem definition:

%    How can we, using a personal voice assistant such as the Amazon \alexa and a smartphone, help citizens who has fallen and cannot get up?
\textit{How can the problem of fall accidents, where the citizen is not able to get up by them self, be assisted by a smart solution, integrating smartphones and personal voice assistance technologies in a web-based solution?}

%To help us answer the question we also construct the following sub-questions:
%\begin{enumerate}
%    \item How can a \textit{smart} solution assist citizen that has fallen but remain conscious?
%    \item How can a \textit{smart} solution assist citizen that has fallen and lost conscious?
%    %\item How can IoT be integrated with the system, to assist in solving the problem?
%\end{enumerate}