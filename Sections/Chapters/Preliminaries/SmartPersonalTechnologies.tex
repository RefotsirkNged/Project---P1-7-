\label{chap:technologies}
In this chapter different smart technologies that can be used for fall detection are examined. In chapter \ref{preliminaries:existingsolutions} an analysis of existing solutions that makes use of specialized devices was presented.

In this chapter the focus is on general technologies that may be found in an ordinary home, looking at what they are, and how they could be integrated in this project.


\section{Smartphone} \label{technologies:smartphone}
\textit{Oxford dictionary} defines a smartphone as:
\begin{figure}[H]
    \begin{center}
        \textit{A mobile phone that performs many of the functions of a computer, typically having a touchscreen interface, Internet access, and an operating system capable of running downloaded apps.} \cite{oxford:smartphone}
    \end{center}
\end{figure}
Some smartphones are designed with elderly people as their target group such as those made by Doro \cite{doro}. Most people also carry their smartphone with them most of the time \cite{smartphonecarry}.

The system could use a smartphone as a general device for detecting and handling fall accidents. Most modern smartphones usually has an accelerometer that could be used for ongoing fall detection. When the accelerometer detects a fall, it could then contact a server, and send information including the position of the citizen. With a microphone it could be possible for the smartphone to listen for activity such as someone yelling while falling. It could also establish a call with someone, such that the fallen citizen will feel more safe being in direct contact with another human being.

\section{Smart Home}
The home is becoming smarter as more things are connected to the internet. This development is often known as the Internet of Things. Light bulbs, refrigerators, coffee machines are all devices that can now be connected to the internet either directly, or through a hub. The devices can then usually be interacted with through the use of a smartphone.

With everything being connected to the internet, an option could be to monitor a citizen, and look for abnormal or no activity. This could also be used as an indication that something has happened.

\section{Personal Assistants}
We define a personal assistant as an intelligent system that is able to understand and interpret meaning in a sentence or text, and follow the command. This is usually done using text or voice commands. It can be something as simple as setting an alarm, or performing a search for information on the internet and then give a quick summary by reading out loud. Most new smartphones come with a personal assistant already installed, while other assistants come on their own devices. Many of the assistants can also be expanded upon by installing new skills the same way apps can be installed on a smartphone.

The system could use a personal assistant as a different way of detecting a falling accident as the personal assistant has no direct means of detecting it. Instead it can listen for a call for help, and then notify others that something has happened.

\section{Conclusion}
There are many different devices that could be used in the home to monitor accidents. This section limits which of the technologies is discussed and used further on in the project.

Smartphones offer many possibilities, and with some models even being targeted at elderly people \cite{doro}, we choose to continue working with this technology.

Smart home devices has a lot of future potential for detection, this technology will be further explored in later parts of the project.

Some personal assistants can also operate on a device separate from the smartphone, and can still be used when the citizen doesn't carry a smartphone.

In the following chapter \ref{problemdefinition}, a problem definition is written, using the limits set by the problem analysis as a whole.