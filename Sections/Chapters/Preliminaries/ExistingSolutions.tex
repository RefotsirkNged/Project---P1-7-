\label{sec:existing_solutions}
In this chapter, some of the current solutions for emergency assistance to citizens are reviewed and some of the important aspects are further analyzed to help expand our knowledge on the subject. This analysis includes devices for emergency assistance and emergency detection.

There is a special focus on what the current solutions can and cannot do, to see if a different solution can be of any benefit to the health care system.

A lot of the existing solutions work in much the same way, based around either a necklace, or an armband on conjunction with a base unit placed somewhere in the house. We have chosen a couple of them to discuss and analyze further. These were chosen based on in which countries they originate, and on what features they contain. 


\section{AutoAlert}
AutoAlert \cite{AAlert} is a system for automatically detecting falling incidents developed in part by Philips. It focuses mainly on a necklace based system, that tries to detect whether or not you have fallen by constantly measuring your movement. Then, when it detects a rapid change in movement, it gives a short delay to cancel the call, and if the citizen does not cancel, it sends out a message to the monitoring center that then calls emergency services.

The necklace also functions as an emergency call button.
This service/product is subscription based, and has different tiers based on what kind of monitoring you need. The two main automatic fall detection services AutoAlert offers is "HomeSafe with AutoAlert" and "GoSafe". They both offer much the same service, except for the fact that "GoSafe" can be taken outside the home. 

This service has some statistics on the website that indicates that the success rate of this product is 95\% \cite{AAlert:homesafe}  detection rate in all cases where there actually was a true fall. This is not necessarily true, as there is no specific links to statistics that prove this point.
This system is well fleshed out and tested over a period of years, and as such is not missing many features. It does however not offer occupancy detectors, as some of the other services does.

The website does not specify battery longevity for the two systems, except for the base unit, which has a battery longevity of about 30 hours.

\section{Sarita}
Sarita Pearl is is still in development. It is a danish solution that tries to be streamlined and have a nice appeal without cutting down on the usability of the product. It still has fall detection, GPS and works over mobile networks. As it is still in development (albeit the later stages) this is subject to change, but currently it does not need a base home unit to use. This It also differs from the previous systems in that it implements an online platform where you can specify safe-areas. When the user of the Pearl leaves the designated area, the system then automatically sends out a warning to the caregivers/monitoring station. The developers tries to make it more appealing by fitting it into a brooch, such that the user can choose their own design. This could be appealing to a certain target group, some might feel more comfortable with something that does not directly look as a medical appliance. As with all the previous solutions, it also has automatic fall detection and corresponding automatic calling for help. 

One of the reasons Sarita could be an improved solution is that especially with dementia affected citizens, Sarita has proven to have a greater acceptance rate. The acceptance rate can be a real problem with some dementia patients, as they simply do not trust new things as easily as a non-dementia citizen.
Sarita has a stated battery life of about 1 week of usage. 


\section{Cekura}
Cekura has a series of comfort alarms that all require some form of user interaction. As with most of the systems it includes a fall sensor version, as well as a selection of other versions of a "help" button. This one is a danish version, while most of the previous ones (except for Sarita) has been from foreign countries. Cekura also offers a talking service, so if the elderly needs to talk to someone (for various reasons, could be feeling sick, could be extreme loneliness etc.) they can push a button and talk to someone about the problem. None of the other systems has clearly stated that they allow this.

\section{TASK}
TASK Systems offers a wider range of products than most of the analysed systems, still mainly focused around emergency situation detection. These products offer both the normal pendant solution, but also a selection of products based around occupancy detection. The difference is mostly in the fact that these are a sort of add-on you can put underneath a pillow, or in a cupboard, to detect activity instead of directly detecting fall incidents. It also has a necklace with automatic fall detection, which is based around (and limited by) a base unit that must be placed in the citizens home. As such TASK is not for citizens who still wish to go outside their homes.

As with all the other services they provide a call-center for 24/7 monitoring and automatic call assistance when the systems actually detect an incident.

The important part about TASK is that it takes a different approach to detecting falling incidents, using technology integrated into furniture and other frequently used parts of a home. They do not offer any specific statistics on how often it works. 

\section{Hvad bliver brugt i det danske sundhedsvæsen i dag (working title) - State of the Art in Denmark}
In this section we will look at an existing solution that has been used by Frederikssund Kommune. The descriptions of the solution can be found in Appendix \ref{appendix:frederikssund:arbejdsgangsbeskrivelse} and \ref{appendix:frederikssund:kvalitetsstandard}. This specific module does not do automatic fall detection, but only provides emergency help in the form of an emergency button.

The purpose of the solution is to allow the citizen an easy way to call for help. The citizens are given a number that they can call for help, or if unable to handle a smartphone, they are instead given an emergency call device to wear around their neck.

Before the citizen can be allowed to call for help, they have to be deemed able to understand when to use it and when not to. This process is described in detail in the aforementioned documents.

The solution is paid by Frederikssund Kommune, but there are also some expenses that the citizen has to pay, such as installation of an electronic lock in the door and sometimes also a land-line phone connection \cite{frederikssund:arbejdesbeskrivelse} \cite{frederikssund:kvalitetsstandard}.

In addition to this specific solution used by Frederikssund Kommune, there also exists a relatively large collection of products presumably used by different danish municipalities.
These can be found at \cite{hmi}. In this database lots of different types of solutions exist, from different emergency call only necklace based systems, to systems based around inactivity or sound detection. Specifically the sound-activated calling system found at \cite{hmilak} is very reminiscent of the type of solution we are trying to implement.

\section{Conclusion}
The devices we have uncovered while researching the existing solutions have mostly been based around either purely emergency calling functionality, or around automatic fall detection. While some products do exist that tries to use sound-based detection this is not the most prominently used solution (at least not in the danish health care system). The sound-based systems we have found while researching the solutions used by the danish healthcare system, was based around sound levels, not around specific voice recognition. As such we believe it might be useful to create a system based around voice recognition.

%https://www.lifeline.ca/en/shop-lifeline/safety-solutions/product-comparison/
%http://www.taskltd.com/telecare-for-falls.html
%https://www.hindawi.com/journals/ijta/2015/576364/
%https://www.sarita.dk/produkt.html#fald-detektion