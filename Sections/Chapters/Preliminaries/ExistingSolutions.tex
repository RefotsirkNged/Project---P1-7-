\label{sec:existing_solutions}
In this chapter, some of the current solutions for emergency assistance to citizens are reviewed. Some aspects are further analyzed. This analysis includes devices for emergency assistance and emergency detection.

The solutions chosen in this analysis were picked out to cover a broad range of systems with the limited time available. The chosen solutions where picked from both the U.S and Denmark, to get a broader perspective on the state of the art solutions.
A second criteria from which the solutions were picked was on what kind of services they offer. furthermore, there is a special focus on which situations the systems handle and which they do not, and which methods they use to handle the different situations.

Some of the existing solutions work similarly, based around either a necklace, or an armband in conjunction with a base unit placed in the house. Some of these has been chosen to discuss and analyze further. These solutions were chosen based on origin country, and their features. 

\section{Sarita}
Sarita Pearl \cite{sarita} is still under development, developed in Denmark. The developers of the system try to make the solution appealing by fitting it into a brooch, and allowing users to choose their own design. It has fall detection, GPS and works over mobile networks. As it is still under development (at the time of writing) this is subject to change, but currently it does not need a base home unit to use. It also differs from the previous systems, in that it implements an online platform where you can specify safe areas. When the user of the Pearl leaves the designated area, the system then automatically sends out a warning to the caregivers. This could be appealing to a certain target group, as some might feel more comfortable with something that does not directly look as a medical appliance. As with all the previous solutions, it also has automatic fall detection and corresponding automatic calling for help. 

One of the reasons Sarita could be a more appealing solution is that especially with dementia affected citizens, Sarita has proven to have a greater acceptance rate \cite{saritaacceptance}. The acceptance rate can be a real problem with some dementia patients, as they simply do not trust new things as easily as people who is does not have the illness. Sarita has a stated battery life of about 1 week. 

\section{Cekura}
Cekura \cite{CekuraFald} has a series of devices that all require some form of user interaction. As with most of the other systems it includes a fall sensor version, as well as a selection of other variations of "help" button products. Cekura is a solution developed in Denmark. Cekura also offers a talking service, that allows a citizen to talk to someone, by pushing a button, which puts them in contact with someone to talk to, about the problem. None of the other systems has clearly stated that they contain this feature.

\section{TASK}\label{task}
TASK Systems offers a wider range of products than the other analyzed systems, but still mainly focused around emergency situation detection. These products offer both a pendant solution, as seen from other products mentioned in this analysis, but also a selection of products based around occupancy detection. The difference is that these are an add-on you can, as an example, put underneath a pillow, or in a cupboard, to detect activity instead of directly detecting fall accidents. It detects inactivity over a period of time, and alerts emergency services based on this. It also has a necklace with automatic fall detection, which is based around (and limited by) a base unit that must be placed in the citizens home. As such TASK is not for citizens who still wish to go outside their homes.

As with the other services they provide a call-center for 24/7 monitoring and automatic call assistance when the system detects a fall accident.

The important part about TASK is that it takes a different approach to detecting falling incidents, using technology integrated into furniture and other frequently used parts of a home. They do not offer any specific statistics on how often it works.

\section{AutoAlert}\label{sec:lifeLine} 
AutoAlert \cite{AAlert} is a system for automatically detecting falling accidents developed in part by Philips. It focuses on a necklace based system, that tries to detect whether or not you have fallen by constantly measuring the users movement. When it detects a rapid change in movement, it gives a short delay to cancel the call, and if the citizen does not cancel, it sends out a message to the monitoring center that then calls emergency services.

The necklace also function as an emergency call button. This service is subscription based, and has different tiers based on what kind of monitoring you need. The two main automatic fall detection services offered by AutoAlert is: "HomeSafe with AutoAlert" and "GoSafe". They both offer similar services, except for the fact that "GoSafe" can be taken outside the home. 

The service "HomeSafe with AutoAlert" has some statistics on their website, which indicates that the success rate of this product is 95\% \cite{AAlert:homesafe} detection rate in all cases where there has been a fall accident. There is how ever no specific links to statistics that prove this point. This system is well fleshed out and tested over a period of years, and as such is not missing many features. It does however not offer occupancy detectors, as TASK\ref{task} does, as some of the other services does.

The website does not specify battery longevity for the two systems, except for the base unit, which has a battery longevity of about 30 hours.

\section{Example Danish Municipality}
This section looks at an existing solution that has been used by Frederikssund Kommune. The descriptions of the solution can be found in Appendix \ref{appendix:frederikssund:arbejdsgangsbeskrivelse} and \ref{appendix:frederikssund:kvalitetsstandard}. This specific module does not do automatic fall detection, but only provides emergency help in the form of an emergency button.

The purpose of the solution is to allow the citizen an easy way to call for help. The citizens are given a number that they can call for help, or if unable to handle a smartphone, they are instead given an emergency call device to wear around their neck.

Before the citizen should be allowed to get a system allocated to them, they have to be deemed able to understand when to use it and when not to. The usual way this process works in Frederikssund Kommune, is described in detail in the aforementioned documents.

The solution is paid by Frederikssund Kommune, but there are also some expenses that the citizen has to pay, such as installation of an electronic lock in the door and sometimes also a land-line phone connection \cite{frederikssund:arbejdesbeskrivelse} \cite{frederikssund:kvalitetsstandard}.

In addition to this specific solution used by Frederikssund Kommune, there also exists a relatively large collection of products presumably used by different danish municipalities.
These can be found at \cite{hmi}. The database contains information about different types of solutions that already exists. The database information ranges from different emergency call only necklace based systems, to systems based around inactivity or sound detection. Specifically the sound activated calling system found at \cite{hmilak} is special, as its solution varies from other solution seen elsewhere in this analysis.

\section{Conclusion}
The different solutions analyzed in this section has mostly been based around either emergency calling functions only, or around automatic fall detection - with some of the solutions doing both.\\
Cekura specifically offered a service that allows the elderly to push a button and then get in contact with someone they can talk to - even though they might not have had an actual falling accident. This is not something the report will further focus on, as that would require an actual calling center - something that is not in the scope of this solution.\\
The solution provided by Task incorporates different connected sensors that detect activity. The important part about Task is that it offers a different approach to fall detection than the other solutions - relying on indirect detection rather than direct detection.\\
The solution provided by Frederikssund kommune is not radically different from the other solutions, these are however still usefull for this analysis, since they gave an idea of what is actually used in Denmark (at the time of writing).

The sound based systems found while researching the solutions used by the danish healthcare system, was based around sound levels, not around specific voice recognition. Furthermore, a lot of the systems are very closed with regards to which devices they allow. As such we believe it might be useful to create a system based around voice recognition, and in general a system that is flexible with regards to the devices it can utilize.

The four cases chosen in \ref{preliminaries:problemanalysis:limitproblem} are fulfilled to various degrees by the solutions analyzed in this section. The solutions that require manual input, will not solve cases 3 and 5. Otherwise, the rest of the cases are solved by most of the above solutions.