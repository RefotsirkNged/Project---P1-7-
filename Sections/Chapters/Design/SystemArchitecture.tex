\label{chap:system-arch}
To better define the architecture of the system, we have chosen to define the class diagrams using UML diagrams. Before presenting the different diagrams, it is important to have a better understanding of the different components that makes up the system. 

After the components has been defined, we evaluate the non-functional requirements for them. This has been done using different system quality criteria.

\section{Components}
Based on the arguments of chapter \ref{chap:technologies}, the program has been split into different component, which have different responsibilities. Since the solution should implement different technologies all running on different platforms, it was decided to connect it all through a Web-Service. This would make it possible to easily connect new components to the solution which expand on the functionality of the solution. In this section the different components, including the Web-Service, have been described and in the end defined with use of class diagrams.

\subsection{Smartphone app}

The functionality of the smartphone app has been separated into two components. The first component is the one used by a citizen to activate an alarm when a fall accident has happened. As mentioned in section \ref{moscow} we have not implemented automatic fall detection, so the app will work by letting the citizen activate the alarm manually. 

When activating an alarm, the citizen should still have a chance to cancel the alarm in case it was a false positive. The second component is used by the contacts. When a citizen has raised an alarm, and the system tries to contact them, they receive a notification. They can then answer, and let the system know if they are able to help or not. If the contact does not answer, the request for help will be invalidated.

All interaction for the app for both components happens through the Web-API, where login, and activation and updating of the alarm happens. The app should not have any functionality when the user isn't logged in.

\subsection{Personal assistant}
The personal assistant will allow the citizen to call for help, by using a single or a few keywords. These words should be clear and logical for the user, but not allow for a false positive, that could occur during normal conversation. When the personal assistant is confident that the citizen has had a falling accident, it notifies the web service. When the request has been acknowledged and a contact has answered, the personal assistant informs the citizen about this and tells the citizen who has responded to the alarm.

\subsection{Web-service}
The Web-Service is the central part of the system, where all computations are done and all communication between the different components happen through it. The web service is accessible though a web-API. When the web-service receives an alarm from a citizens device, it is responsible for making contact to the different contacts associated with the citizen that has had the fall accident, and report back to the citizen when a contact has responded to the alarm.

The web-API used to access the functionality of the web-service, can be used by any device, as long as they uses the correct structure for the API calls.

\subsection{Control panel}
The control panel is used to manage users and their associated information. From it, it is possible to create new citizens, define their contact-list, devices, and manage their login information. It should be separated from the rest of the system, as to avoid confusion that could lead to changes in settings or the like.

\section{System criteria}
\label{sec:non-requirements}
The criteria of the system has been defined for each component. These work as guidelines during the design of the system architecture. The different criteria that is used is from the book \textit{Objekt orienteret analyse \& design} \cite{subook}.

The criteria for each component are as seen on figure \ref{tab:non-functional}, each decision has been elaborated on further in this section.



\begin{table}[H]
\centering
\begin{tabular}{|l|c|c|c|c|}
\hline
              & \multicolumn{1}{l|}{Smartphone app} & \multicolumn{1}{l|}{Personal assistant} & \multicolumn{1}{l|}{Web-Service} & \multicolumn{1}{l|}{Control panel} \\ \hline
Usable        & x                                   & x                                &                              & x                                  \\ \hline
Secure        &                                     &                                  & x                            & x                                  \\ \hline
Efficient     & x                                   &                                  & x                            &                                    \\ \hline
Correct       & x                                   & x                                & x                            & x                                  \\ \hline
Reliable      & x                                   & x                                & x                            &                                    \\ \hline
Maintainable  &                                     &                                  & x                            &                                    \\ \hline
Testable      & x                                   & x                                & x                            &                                    \\ \hline
Flexible      &                                     &                                  & x                            &                                    \\ \hline
Comprehensive &                                     &                                  &                              &                                    \\ \hline
Reuseable     &                                     &                                  &                              &                                    \\ \hline
Portable      & x                                   &                                  &                              &                                    \\ \hline
Interoperable &                                     &                                  & x                            &                                    \\ \hline
\end{tabular}
\caption{System criteria}
\label{tab:non-functional}
\end{table}


\paragraph{Usable}
Usability is an important aspect for our solution, since it will be used by elderly citizens that often do not have much experience with technology. 
The control panel should be simple enough so that the citizen-admins easily can manage the citizens, contacts and devices.

The smartphone app should be simple to make it easier for the elderly to use in case they have fallen. 

The personal assistant will be talking with the elderly citizen, the interaction between these should be simple, such that the elderly citizen can manage the conversation, even in a state of panic that might occur during or after a fall accident.

\paragraph{Secure}
It is important for the Web-Service and control panel to be secure as they may allow access to personal information for the citizens.

Since the smartphone app and the personal assistant would not have access to that data anyway there is no need to think about making them more secure than they are by the manufacturer.

\paragraph{Efficient}
The efficiency of the smartphone is important due to concerns about not draining the battery. The app would be constantly running, either in the background or normally, and could therefore easily be draining the battery if not implemented in an efficient manner.

The efficiency of the Web-Service is important due to the possibly large amount of requests being made at any given time. If the system ends up having a lot of users, this would be important.

\paragraph{Correct}
Every part of the system needs to fulfill the specified requirements.

\paragraph{Reliable}
The smartphone app needs to be able to perform its fall-detection precisely enough that it does not create either false positives or false negatives.

The personal-assistant needs to be able to perform its voice recognition precisely enough that it does not create either false positives or false negatives. In this case this would mean being able to detect a cry for help well enough that it does not detect a normal conversation as one.

The Web-Service needs to be precise enough that it does not corrupt data. This could for example be that it does not give out the wrong data when a request is made for user-data, or send out wrong notifications regarding falling accidents. It also should never lose data.

\paragraph{Maintainable}
The Web-Service needs to be maintainable enough that it is possible to add future IoT devices to it without it requiring a rewrite of the Web-Service every time an addition or removal of an IoT device is needed.

\paragraph{Testable}
It is important that the smartphone app, personal assistant and Web-Service is thoroughly tested, because if the components malfunction when they are needed, the citizen in need of help will not get it.

\paragraph{Flexible}
The Web-Service needs to be flexible enough that it is possible, and economically feasible to change parts of the Web-Service after it has been taken into use.

\paragraph{Comprehensive}
The system is so simple that it will not make a noticeable difference if it is comprehensive or not.
%No part of the system absolutely needs to be comprehensive, as long as the system is usable by the user.

\paragraph{Reuseable}
There will be no focus on making any of the components in the system reusable.

\paragraph{Portable}
The smartphone app will need to be written in a way that makes it portable, since it will need to work on the smartphone that the citizen has. 

\paragraph{Interoperable}
The Web-Service needs to be able to integrate with new devices without it being too big of a cost. This enables possible future extension of the system, with various new IoT devices.
