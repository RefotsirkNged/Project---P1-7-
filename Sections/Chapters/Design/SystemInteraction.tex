In this chapter we will look at different ways to detect when citizens fall. We will look at two different ways to monitor it: directly and indirectly. We limit ourselves to looking at sensors for the direct and the use of voice for the indirect, as that will allow us to handle both situations where the user is and is not conscious.

\section{Physical Fall Detection}
The most direct way to detect if a citizen has fallen is by using some kind of device containing different sensors. This device can either be some embedded device, or a more advanced device such as a smartphone.\\
A general drawback for all worn devices is that they run on battery, and thus also have to be charged to be of any use for the citizen.

\subsection{Embedded devices}
An embedded system could be custom designed to this use case. As said in \ref{sec:existing_solutions}, there are already solutions using different variations of embedded devices, hence they are an option to consider.

\subsection{Smartphone}
Most smartphones has an accelerometer, that can be used to detect the movement. Many elders do not know how to use a smartphone, or may find themselves feeling uncomfortable with it. But since it is an available and plentiful \todo{er smartphones en resource?}resource, it is a great tool for prototyping physical fall detection. A system always watches and monitors the user may however use a lot of power, so the longitivity may be limited.

\subsection{Smartwatch}
An in-between solution could be a smartwatch. The smartwatch also has an accelerometer and a GPS antenna making it a viable device. A smartwatch is designed to be power efficient, and therefor maybe has a longer longevity compared to smartphones.


\section{Voice Monitoring}
Physical fall detection may not be accurate enough to always detect a fall, so another way of detecting a fall will be needed. By using voice, it is possible to listen for when the user is calling for help.
Many existing solutions make use of buttons in accessories, but this section will not go in details as that can be read in section \ref{sec:existing_solutions}.

\subsection{Manual Call}
The simplest way to listen for help is letting the fallen citizen make the call. This is used by some of the solutions mentioned in chapter \ref{preliminaries:existingsolutions}. This require the user to be able to access and use a phone. Using this solution requires the citizen to actively perform an action when they have fallen in order to call for help.

\subsection{Voice Monitoring With Smartphone}
All smartphones have a microphone, which can be used to listen for when the citizen calls for help. Apps can get access to the microphone, and always listen for input with the users permission. This would however require a library for voice recognition, running the microphone all the time and processing the voice recognition takes a lot of power. If a smartphone were to be used for voice monitoring, it would be necessary to extend the smartphone with devices designed to constantly listen, such as the Personal Assistants described in \ref{preliminaries:systeminteraction:personalassistant}. They are compatible on the big mobile operating systems iOS, Android and Windows Phone.

\subsection{Personal Assistant}\label{preliminaries:systeminteraction:personalassistant}
A personal assistant is an intelligent piece of software that the user is able to interact with by using voice commands or text messages in a natural way, that can interpret the input and perform an action based on it. This can include gathering info about the current weather, make a note, set an alarm, or fetch information from the internet. Some of them even allows adding custom functionality.\\
We will look at the personal assistants provided by the big companies that people encounter in their daily lives.

\subsubsection{Amazon Alexa}
\alexa is a personal assistant developed by Amazon, that runs on the Amazon Echo devices, as well as smartphones. \alexa works by listening for the command \textit{Alexa} followed by the name of the command. The apps running on \alexa are called Skills. A skill can have many different ways of invoking the same command or version of the same command, since the developer can specify each way \alexa should recognize commands for the specific skill \cite{AmazonAlexa, AmazonAlexa:UnderstandingCustomSkills}.\\
It is possible to add custom skills to \alexa, that can be developed using the Amazon Developer Console for free.

\subsubsection{Apple Siri}
\siri is a personal assistant developed by Apple, that runs on Apple iOS and OSX devices, including iPhone, Mac, iPad, and AppleTV \cite{AppleSiri}. It is possible to integrate Siri in apps on Apple devices by using the SiriKit \cite{AppleSiri:Sirikit}.

\subsubsection{Microsoft Cortana}
\cortana is a personal assistant developed by Microsoft, that runs on Windows and Windows Phones\cite{MicrosoftCortana}. Microsoft has a development kit for developing functionality, also called skills, for Cortana \cite{MicrosoftCortana:SetUpCommands}.

\subsubsection{Samsung Bixby}
\bixby is a personal assistant developed by Samsung running on their smartphone Samsung Galaxy 8 and newer \cite{SamsungBixby}. Unlike the other personal assistants, \bixby cannot be used with custom commands. It is possible to create new voice commands, but they are limited to interacting with existing functionality.

\subsubsection{Google Assistant/Now}
Google has two different personal assistants, Google Now and Google Assistant. Google Now is the older of the two running in Android and iOS apps, while Google Assistant is newer and more powerfull as it has access to Googles natural language processing engine \cite{GoogleNowAssistant}. Google Assistant can be set up with IFTTT to create custom voice commands to control services \cite{GoogleNowassistant:IFTTT}.  

\subsubsection{Facebook M}
M is a personal assistant the runs inside Facebook Messenger. It is possible for the user to use M directly from the chat to do most of the same things as other personal assistants, and for some actions there may be a human behind to complete the task. Right now, M is limited to the United States, and with no options of expanding on it beyond what Facebook adds to it\cite{FacebookM}.

\subsection{Voice Monitoring With Anything}
In addition to the \alexa Amazon also provides a tool called AVS, which stands for Alexa Voice Service. This service brings the voice control of the \alexa into anything. 


\section{Conclusion}
We choose to use \alexa as our personal assistant. This choice is made because it can run on the Amazon Echo, which is an external device made for speaking out to in an open room. It also does not limit the user to carrying a smartphone of a specific brand, or having a full computer running at all time. In addition to that it is free to develop new skills.\\
We will still use a smartphone as supplement, because it gives us easy access to an accelerometer and platform to connect to the internet.
